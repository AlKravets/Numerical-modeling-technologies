\documentclass[a4paper,12pt]{article}
\usepackage[T1,T2A]{fontenc}
\usepackage[utf8]{inputenc}
\usepackage[english,ukrainian]{babel}

\usepackage{ upgreek }
\usepackage{amsmath}

\usepackage{graphicx}
\graphicspath{{../pictures/}}
\DeclareGraphicsExtensions{.pdf,.png,.jpg}

\usepackage{hyperref}
\hypersetup{colorlinks = true}
\usepackage{geometry} % Меняем поля страницы
\geometry{left=2cm}% левое поле
\geometry{right=1.5cm}% правое поле
\geometry{top=1cm}% верхнее поле
\geometry{bottom=2cm}% нижнее поле



\begin{document}
	
	\begin{titlepage}
		\vspace*{6cm}
		\begin{center}
			
			\large
			\textbf{Звіт}\\
			\textbf{до лабораторної роботи}\\
			\textbf{<<Застосування
				методу дискретних особливостей для\\
				моделювання аеродинамічних процесів та обчислення поля тисків>>}
			
		\end{center}
		
		\vspace{8cm}
		\begin{flushright}
			студента 2-го курсу магістратури\\
			факультету комп'ютерних наук та кібернетики\\
			Кравця Олексія
		\end{flushright}
		
	\end{titlepage}
	
	\newpage
	
	\section{Постановка задачі}
	
	
	Дано контур $L_d$, що знаходиться в області $D$, з течією на нескінченості $V_{\infty} = (u_{\infty}(t), v_{\infty}(t))$. Позначимо через $L_v$ вільну границю. Також вважаємо, що течія безвихорова, тобто $\exists \varphi = \varphi(x,y,t) : \vec{V} = \nabla \varphi$. Потенціал $\varphi$ в області $D$ задовольняє рівняння Лапласа: $\Updelta \varphi =0$.
	
	Крім того на контурі $L_d$ виконується умова непроникності:
	\[
	\left( \nabla \varphi \cdot \vec{n} \right)|_{L_d} = 0,
	\]
	де $\vec{n}$ -- нормаль до поверхні $L_d$.
	
	На вихровому контурі $L_v$ виконуються такі умови:
	\begin{align}
	\left( \frac{\partial \varphi}{\partial \vec{n}} \right)^+ &= \left( \frac{\partial \varphi}{\partial \vec{n}} \right)^- \quad \text{на } L_v \nonumber \\
	p^+ &= p^- \quad \text{на } L_v \nonumber
	\end{align}
	
	Також вважаємо, що $\lim_{|r| \rightarrow \infty} \nabla \varphi = \vec{V}_{\infty}$, тобто із нескінченності набігає потік сталої швидкості. Також вважаємо, що швидкість скінченна $|\nabla \varphi| < \infty$ на гострих кутах $L$.
	
	Інтеграл Коші-Лагранжа:
	\[
	\frac{\partial \varphi}{\partial t} + \frac{1}{2}\left( \nabla \varphi \right)^2 + \frac{p}{\rho} = \frac{p_{\infty}}{\rho} + \frac{1}{2} V_{\infty}^2 + \frac{\partial \varphi_{\infty}}{\partial t}
	\]
	Інтегральне представлення аналітичного розв'язку:
	\begin{align}
	&\Phi(z,t) = \varphi + i\xi = \vec{V}_\infty z + \frac{1}{2 \pi i} \int_{L_d(t)} f(w,t) \ln{(z - w)}dw + \frac{1}{2 \pi i} \int_{L_v(t)} f(w,t) \ln{(z - w)}dw + const \nonumber \\
	&\vec{V}(z,t) = u- iv = \frac{\nabla \Phi(z,t)}{\nabla z} = \vec{V}_\infty + \frac{1}{2 \pi i} \int_{L_d(t)} \frac{f(w,t)}{z - w} dw + \frac{1}{2 \pi i} \int_{L_v(t)} \frac{f(w,t)}{z - w} dw \nonumber \\
	&C_p(x,y,t) = 2 \frac{p - p_\infty}{\rho \vec{V}_\infty^2} = 1 - \frac{(\nabla \varphi)^2}{\vec{V}_\infty^2} - \frac{2}{\vec{V}_\infty^2} \frac{\partial \varphi}{\partial t} \nonumber
	\end{align}
	\section{Моделювання кінематики}
	
	Необхідно дискретизувати контур, розіб'ємо його на $M$ точок $(x_{0j}, y_{0j}), j = \overline{1,M}$. Тепер знайдемо точки колокації:
	\begin{eqnarray}
	x_k = \frac{x_{0k} + x_{0, k+1}}{2} \nonumber \\
	y_k = \frac{y_{0k} + y_{0, k+1}}{2} \nonumber
	\end{eqnarray}
	
	Також проведемо нормалі в точках колокацій:
	
	\begin{align} 
	\overrightarrow{n_k} (x_k, y_k) = (n_{xk}, n_{yk}),\quad  k = \overline{1, M-1} \nonumber \\
	n_{xk} = \frac{-(y_{0,k+1} - y_{0k})}{\sqrt{(x_{0, k+1} -  x_{0k})^2 + (y_{0,k+1} - y_{0k})^2}} \nonumber \\
	n_{yk} = \frac{x_{0, k+1} -  x_{0k}}{\sqrt{(x_{0, k+1} -  x_{0k})^2 + (y_{0,k+1} - y_{0k})^2}} \nonumber
	\end{align}
	
	\begin{figure}[ht]
		\begin{center}\includegraphics[scale=1]{barrier_plot} \end{center}
		\caption{Вигляд контуру}
		\label{fig:barrier_plot}
	\end{figure}
	
	На рисунку ($\ref{fig:barrier_plot}$) бачимо вигляд контуру. На ньому позначені точки дискретних особливостей, точки колокацій та точки відриву, з яких утворюються вихори.
	
	Тепер розв'яжемо задачу чисельно. Вважаємо, що модуль швидкості $|\vec{V}_\infty| = 1$, тобто $V_\infty = (\cos{\alpha}, \sin{\alpha})$. Для обчислення потенціалу і швидкості в момент часу $t = t_{n+1}$ будемо використовувати наступні формули:
	\newpage
	
	\begin{align}
	\varphi(x,y,t_{n+1}) &= (x \cos \alpha + y \sin \alpha) + \sum\limits_{j=1}^{M} \frac{\Gamma_j(t_{n+1})}{2 \pi} Arctg \left(\frac{y - y_{0j}}{x - x_{0j}}\right)  \nonumber \\ &+\sum_p \sum\limits_{i=1}^{n+1} \frac{\gamma_{i}^{p}}{2 \pi} Arctg \left( \frac{ y - y_{i}^{p}(t_{n+1}) }{ x - x_{i}^{p} (t_{n+1}) }\right) \nonumber	
	\end{align}
	
	\begin{align}
	\vec{V}(x,y,t_{n+1}) &= (\cos \alpha, \sin \alpha) + \sum_{j=1}^{M} \Gamma_{j}(t_{n+1}) \vec{V}(x,y,x_{0j}, y_{0j}) \nonumber \\ &+\sum_p \sum_{i=1}^{n+1} \gamma_{i}^{p} \vec{V}(x,y,x_{i}^{p}(t_{n+1}), y_{i}^{p}(t_{n+1})) \nonumber
	\end{align}
	
	Перед тим, як почати розв'язувати ці рівняння, введемо декілька величин.
	\begin{equation*}
	R_{0i} = 
	\begin{cases}
	\sqrt{ (x - x_{0i})^2 + (y - y_{0i})^2 }, &\sqrt{ (x - x_{0i})^2 + (y - y_{0i})^2 } > \delta
	\\
	\qquad \delta, &\sqrt{ (x - x_{0i})^2 + (y - y_{0i})^2 } \le \delta
	\end{cases}
	\end{equation*}
	
	\begin{equation*}
	\vec{V}(x,y,x_{0i}, y_{0i}) =
	\begin{cases}
	u(x,y,x_{0i}, y_{0i}) = \frac{1}{2 \pi} \frac{y_{0i} - y}{R_{0i}^2}
	\\
	v(x,y,x_{0i}, y_{0i}) = \frac{1}{2 \pi} \frac{x - x_{0i} }{R_{0i}^2}
	\end{cases}
	\end{equation*}
	
	Для того, щоб знайти коефіцієнти $\Gamma_{j}$ необхідно розв'язати систему лінійних алгебраїчних рівнянь.
	
	\begin{equation*}
	\begin{cases}
	\sum_{j=1}^{M} \Gamma_{j}(t_{n+1})\left( \vec{V}(x_k,y_k, x_{0j}, y_{0j}) \cdot \vec{n}(x_k,y_k) \right) = - \left[ \left(  \vec{V}_{\infty} \cdot \vec{n}(x_k, y_k) \right) \right.
	\\
	+ \left. \sum_p \sum_{i=1}^{n+1} \gamma_{i}^{p} \left( \vec{V}(x_k,y_k,x_{i}^{p}(t_{n+1}), y_{i}^{p}(t_{n+1})) \cdot \vec{n}(x_k, y_k) \right)\right], k = \overline{1,M-1}
	\\
	\sum_{j=1}^{M} \Gamma_{j}(t_{n+1}) = - \sum_p \sum_{i=1}^{n+1} \gamma_{i}^{p}
	\end{cases}
	\end{equation*}
	
	Одночасно з цим проходить оновлення точок вихорової границі. Для цього використовуємо метод Ейлера. Тобто $\forall p, i = \overline{1,n}$:
	\begin{equation*}
	\begin{cases}
	x_{i}^{p}(t_{n+1}) = x_{i}^{p}(t_n) + u(x_{i}^{p}(t_n), y_{i}^{p}(t_n), t_n)\tau_n
	\\
	y_{i}^{p}(t_{n+1}) = y_{i}^{p}(t_n) + v(x_{i}^{p}(t_n), y_{i}^{p}(t_n), t_n)\tau_n
	\end{cases}
	\end{equation*}
	де $$\tau_n = \frac{\min_{k} \delta_k}{\max_{D}(|V|)}.$$
	
	На кожному кроці з кінцевих та кутових точок перешкоди вилітають нові вихорові точки. При цьому, вони мають такі властивості:
	\[
	\forall p: \gamma_{n+1}^{p} = \Gamma_p(t_n);
	\]
	\begin{equation*}
	\begin{cases}
	x_{n+1}^{p}(t_{n+1}) = x_{0p}(t_n) + u(x_{0p}(t_n), y_{0p}(t_n), t_n)\tau_n
	\\
	y_{n+1}^{p}(t_{n+1}) = y_{0p}(t_n) + v(x_{0p}(t_n), y_{0p}(t_n), t_n)\tau_n
	\end{cases}
	\end{equation*}
	
	\subsection{Знаходження поля тисків}
	
	Маємо наступну аналітичну формулу:
	\[
		C_p(x,y,t) = 2 \frac{p - p_\infty}{\rho \vec{V}_\infty^2} = 1 - \frac{(\nabla \varphi)^2}{\vec{V}_\infty^2} - \frac{2}{\vec{V}_\infty^2} \frac{\partial \varphi}{\partial t}
	\]
	або це можна записати таким чином
	
	\[
		C_p(x,y,t) = 1 - \frac{\left( \vec{V}(x,y,t) \right)^2}{\vec{V}_\infty^2} - \frac{2}{\vec{V}_\infty^2} \left(\frac{\partial \varphi_{\text{дипол}}}{\partial t} + \frac{\partial \varphi_{\text{конвект}}}{\partial t} \right)
	\]
	
	Чисельно похідну потенціалу $\frac{\partial \varphi}{\partial t}$ можна знайти за такою формулою:
	
	\begin{align}
		&\frac{\partial \varphi(x,y,t_{n+1})}{\partial t} = \sum_{j=1}^{M-1} \left( \vec{D}_j, \vec{V}_j \left(x,y,\overline{x}(t_{n+1}), \overline{y}(t_{n+1}) \right) \right) + \sum_p \left( \vec{d}_p, \vec{V}_p \left(x,y,\overline{x}^p_n(t_{n+1}), \overline{y}^p_n(t_{n+1}) \right)  \right) - \nonumber \\
		&- \sum_{j=1}^M \Gamma_j (t_{n+1}) \left(
			\vec{V}_j \left( x,y,x_{0j}(t_{n+1}), y_{0j}(t_{n+1}) \right), \vec{V} (x_{0j}(t_{n+1}), y_{0j}(t_{n+1}) \right) -  \nonumber \\
		&- \sum_{p=1}^{P}\sum_{i=1}^{n+1} \gamma_i^p \left(
			\vec{V}_i \left( x,y, x_i^p(t_{n+1}), y_i^p(t_{n+1}) \right), \vec{V} (x_i^p(t_{n+1}), y_i^p(t_{n+1})\right) \nonumber
	\end{align}
	де
	
	\[
	\begin{cases}
		\vec{D}_j = (x_{0,j+1} - x_{0,j}, y_{0,j+1 - y_{0,j}}) Q_j \\
		\vec{d}_p = (x_{0,p} - x_{n}^p, y_{0,p} - y_{n}^p)
	\end{cases} \quad
	\begin{cases}
		\overline{x}_j = 0.5(x_{0,j+1} + x_{0,j}) \\
		\overline{y}_j = 0.5(y_{0,j+1} + y_{0,j})
	\end{cases}	\quad
	\begin{cases}
		\overline{x}_n^p = 0.5(x_n^p + x_{0,p}) \\
		\overline{y}_n^p = 0.5(y_n^p + y_{0,p}) 
	\end{cases}	
	\]
	також роз'яснимо позначення:
	
	\begin{align}
		&\dot{\gamma}_{n+1}^p = \frac{\gamma_{n+1}^p}{t_{n+1} - t_n} \nonumber \\	
		&\dot{\Gamma}_j = \frac{\left(\Gamma_j(t_{n+1}) - \Gamma_{j}(t_{n})\right)}{t_{n+1} - t_n} \nonumber \\
		& q_j = \dot{\Gamma}_j, \qquad q_p(t_{n+1}) = \dot{\Gamma}_p(t_{n+1}) + \dot{\gamma}_{n+1}^p \nonumber \\
		&Q_q = q_1, \quad Q_i = \sum_{k=1}^i q_k \quad i = 1,2, \ldots M-1 \nonumber
	\end{align}
	
	\subsection{Забезпечення непроникності контуру $L_d$.}
	
	Для забезпечення непроникності контуру необхідно накласти деякі умови. Відстань між двома точками дискретних особливостей повинна бути меньше за $2\delta$, де $\delta$ -- константа, що використовується для розрахунку $\tau$. Це дає нам змогу відслідковувати ті вихорові точки, що наблизилися до границі.
	
	Нехай маємо точку дискретної особливості $z_0 = (x_0, y_0)$, до якої наблизилася точка вихорової границі $z_1(t_{n+1}) = (x_i(t_{n+1}), y_i(t_{n+1}))$ на відстань менше $2 \delta$.
	
	Необхідно зрозуміти з якої сторони контуру наблизилася вихорова точка. Тому візьмемо точку $z_1(t_{n})$ -- точку де була точка $z_1(t_n+1)$ на минулому кроці, та обрахуємо $\lambda = sign \left( (\vec{n_0}, z1(t_n) - z_0) \right)$, де $\vec{n_0}$-- нормаль в точці $z_0$
	
	Тепер можна обрахувати коефіцієнт $k = \lambda|z_0 - z_1|$. Отже змінимо точку $z_1(t_{n+1}) = z_1(t_n) + k\vec{n_0}$.
	
	\section{Результати}
	\subsection{Перешкода у вигляді пластинки}
	Розглянемо результати при $V_{\infty} = 1 + 0i$.
	
	\begin{figure}[ht]
		\begin{center}\includegraphics[scale=0.45]{plate_30} \end{center}
		\caption{Результат при швидкості $V_{\infty} = 1+0i$}
		\label{fig:result_plate1}
	\end{figure}
	
	Збільшимо кількість кроків
	
	\begin{figure}[]
		\begin{center}\includegraphics[scale=0.45]{plate_100} \end{center}
		\caption{Результат при швидкості $V_{\infty} = 1 +0i$}
		\label{fig:result_plate2}
	\end{figure}
	\newpage
	
	\subsection{Перешкода у вигляді літери U}
	
	Розглянемо результати при $V_{\infty} = 1 +0i$.
	
	\begin{figure}[h]
		\begin{center}\includegraphics[scale=0.5]{U_form_30} \end{center}
		\caption{Результат при швидкості $V_{\infty} = 1 +0i$}
		\label{fig:result_one1}
	\end{figure}

	Збільшимо кількість кроків
	
	\begin{figure}[ht]
		\begin{center}\includegraphics[scale=0.5]{U_form_100} \end{center}
		\caption{Результат при швидкості $V_{\infty} = 1 +0i$}
		\label{fig:result_one2}
	\end{figure}

	
	\newpage
	\subsection{Перешкода у вигляді цифри 3}
	
		\begin{figure}[hb]
		\begin{center}\includegraphics[scale=0.45]{form_3_100} \end{center}
		\caption{Результат при швидкості $V_{\infty} = 1 +0i$}
		\label{fig:result_3}
	\end{figure}
	

	\newpage
	\section{Висновок}
	Було змодельовано задачу обтікання заданого непроникного контура. Для розв'язання даної задачі було використано метод дискретних особливостей. Було побудоване поле тисків. Можливі помилки можуть бути спричинені помилками при обробці непроникності границі перешкоди.
	
\end{document} 
